\chapter*{Resumen}

La visi�n computacional y la atenci�n visual son hoy en d�a uno de los campos m�s
atractivos e interesantes en los que la comunidad de investigaci�n
dedica tiempo, esfuerzo y dinero para avanzar. Las
c�maras, cual ojos en el ser humano pueden proveen de informaci�n
sensorial important�sima sobre el entorno a los robots. El problema es
que extraer informaci�n de una imagen no es algo trivial, ya que hay
que distinguir las partes importantes e interesantes de la
imagen. Para ello se utilizan sistemas de atenci�n visual que nos
ayudan a filtrar lo relevante de lo desapercibido, analizando el flujo
de datos de las im�genes lo m�s r�pido posible.


El presente proyecto aborda el desarrollo para un robot m�vil de un
sistema de atenci�n visual unido a una representaci�n en 3D del
entorno. Para ello se realiza un control atentivo, basado en un mapa
de saliencia, sobre los objetos relevantes de la imagen. Por medio de
un par de c�maras, y analizando las im�genes reales, localizamos esos
objetos relevantes en el mundo 3D. El sistema atentivo programado
analiza la imagen de una forma similar a como lo hace un ojo humano,
fij�ndose en las cosas que suelen llamar la atenci�n: colores
llamativos, bordes, movimiento o conocimiento previo de los objetos en
la realidad. Ciertas teor�as, argumentan que el sistema visual humano tiene
una capacidad de procesamiento limitado y que la atenci�n act�a como
un filtro neuronal que selecciona la informaci�n debe ser procesada en
cada instante.


Se ha puesto �nfasis en la velocidad de proceso del sistema. Es
necesario que se realice la atenci�n visual sobre las im�genes reales
y la representaci�n 3D en el menor tiempo posible, para que el sistema
tenga un comportamiento parecido al de un ser humano. A lo largo de la
memoria se comentar�n los numerosos ajustes que se han llevado a cabo
para obtener un rendimiento aceptable.


Para la implementaci�n de este proyecto, se ha utilizado la arquitectura
y plataforma software de JDE, programando una serie de esquemas en C y
C++. Adem�s se han integrado y mejorado varios esquemas y librer�as de
esta infraestructura, obteniendo un correcto y eficiente funcionamiento.
